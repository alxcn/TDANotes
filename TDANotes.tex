 % Load the kaobook class
\documentclass[
	fontsize=10pt, % Base font size
	twoside=false, % Use different layouts for even and odd pages (in particular, if twoside=true, the margin column will be always on the outside)
	%open=any, % If twoside=true, uncomment this to force new chapters to start on any page, not only on right (odd) pages
	secnumdepth=1, % How deep to number headings. Defaults to 1 (sections)
]{kaobook}

\usepackage[english]{babel} % Load characters and hyphenation
\usepackage[english=british]{csquotes}	% English quotes
\usepackage{kaobiblio}
%\addbibresource{minimal.bib}
\usepackage{kaotheorems}

% Load the package for hyperreferences
\usepackage{kaorefs}

\graphicspath{{images/}{./}} % Paths where images are looked for

\makeindex[columns=3, title=Alphabetical Index, intoc]
\begin{document}

%----------------------------------------------------------------------------------------
%	BOOK INFORMATION
%----------------------------------------------------------------------------------------

\titlehead{A manual for MA2007B}
\title[Applications of Geometry and Topology for Data Science]{Applications of Geometry and Topology for Data Science}
\author[AUP]{Alejandro Ucan-Puc}
\date{\today}
\publishers{Instituto Tecnólogico y de Estudios Superiores de Monterrey}

\maketitle
%----------------------------------------------------------------------------------------

\frontmatter % Denotes the start of the pre-document content, uses roman numerals
%----------------------------------------------------------------------------------------
%	PREFACE
%----------------------------------------------------------------------------------------

%\chapter*{Preface}


%----------------------------------------------------------------------------------------
%	TABLE OF CONTENTS & LIST OF FIGURES/TABLES
%----------------------------------------------------------------------------------------

\begingroup % Local scope for the following commands

% Define the style for the TOC, LOF, and LOT
%\setstretch{1} % Uncomment to modify line spacing in the ToC
%\hypersetup{linkcolor=blue} % Uncomment to set the colour of links in the ToC
\setlength{\textheight}{230\vscale} % Manually adjust the height of the ToC pages

% Turn on compatibility mode for the etoc package
\etocstandarddisplaystyle % "toc display" as if etoc was not loaded
\etocstandardlines % "toc lines as if etoc was not loaded

\tableofcontents % Output the table of contents

\listoffigures % Output the list of figures

% Comment both of the following lines to have the LOF and the LOT on different pages
\let\cleardoublepage\bigskip
\let\clearpage\bigskip

\listoftables % Output the list of tables

\endgroup

%----------------------------------------------------------------------------------------
%	MAIN BODY
%----------------------------------------------------------------------------------------

\mainmatter % Denotes the start of the main document content, resets page numbering and uses arabic numbers
\setchapterstyle{kao} % Choose the default chapter heading style

\chapter{Introduction to Topology}

%\pagelayout{wide} % No margins
%\addpart{aasja}
%\pagelayout{margin} % Restore margins

Topology was first studied in...

\section{Basic concepts and Examples}

Topology is the study of properties that are preserved under continuous functions in a generic way, for this reason, we need to work with objects that are proper to the space and these objects are subsets. The set theory is needed to properly understand topology because it is the main language of the theory, if you need a refresh of set theory you can see the reference.

\begin{definition}[Topological Space]
Let $X$ be a set and $\tau$ a collection of subsets of $X.$ We say that $\tau$ is a \emph{topology} for $X$ is it satisfies:
\begin{enumerate}
\item If the total and the empty set belong to $\tau.$  $$\varnothing, X\in\tau.$$

\item For every subcollection of elements of $\tau,$ let say $\{U_\alpha\}_{\alpha\in A},$ we have that its union is an element of $\tau.$ $$\bigcup_{\alpha \in U} U_\alpha \in \tau.$$

\item For every finite sub-collection of elements of $\tau,$ let say $\{U_j\}_{j=1}^n,$ we have that its intersection is an element of $\tau.$ $$ \bigcap_{j=1}^n U_j\in\tau.$$
\end{enumerate}

The elements of $\tau$ are called \emph{open sets} of $X$ and the pair $(X,\tau)$ is called a \emph{topological space}.
\end{definition}

This is not the first time that we see the concept of openness, for example in calculus we already worked with open intervals, open boxes (the real plane or space), and open disks (complex numbers). Topology theory intends to generalize this concept to general sets.

\begin{example}[Trivial topology]
Let $X$ be any set and consider the collection $\tau=\{\varnothing, X\}.$ $\tau$ is a topology for $X$ and it is called the trivial topology. 

This is not the best topology to understand properties of $X$ but it is a good counterexample when you want to generalize topological constructions.
\end{example}

\begin{example}[Discrete topology]
Let $X$ be any set and consider the collection $2^X=\{\mbox{all possible subsets of } X\}.$ $\tau$ is a topology for $X$ and it is called the discrete topology. 

Note that each element of $X$ is an open set on this topology. 
\end{example}

\begin{example}[Co-finite topology]
Let $X$ be any set and consider the collection $\tau_{cf}=\{U\subset X: X\setminus U \mbox{ is finite or }\varnothing\},$ i.e., the complement of $U$ in $X$ is a finite set of points. $\tau$ is a topology for $X$ and it is called the co-finite topology. 
\end{example}


\begin{example}[Spaces with different topologies]
Let $X=\{1,3,5,7\}$ and consider the collections

\begin{eqnarray*}
\tau_1 &=& \{\varnothing, \{1\},\{5\},\{7\},\{1,5\},\{1,7\},\{5,7\},\{1,5,7\},X\}. \\
\tau_2 &=& \{\varnothing, \{1\},\{3\},\{7\},\{1,3\},\{1,7\},\{3,7\},\{1,3,7\},X\}.
\end{eqnarray*}

Both $(X,\tau_1)$ and $(X,\tau_2)$ are topological spaces but they are not the same space. For example, the open set $\{1,5\}$ is not part of $(X,\tau_2)$ and $\{1,3\}$  is not part of $(X,\tau_1).$ 

Therefore in order to obtain the same topological space we need to find exactly the same open sets in both topologies.
\end{example}

The next example states a definition that will be useful in future chapters (see Simplicial Complexes and Persistent homology). 

\begin{example}[Graphs topology]
A \emph{graph} is a combinatorial object that consists of two sets:
\begin{itemize}
\item $V$ called vertices, and usually you can consider it as a set of points.
\item $E$ called edges, it is a subset of $V\times V$ (pairs of elements in $V$). This set describes how to vertices are related.
\end{itemize}
 
 Usually, graphs are denoted as $\Gamma=(V, E).$ Given a graph, you can associate a geometric object as a set of points in $\mathbb{R}^n$ in identification with $V$ and lines joining this set of points with the information of $E.$ This geometric object is called the \emph{geometric realization}.
 
% \begin{figure}[h]
 %put examples of graphs 
 %\end{figure}
 
 For example, we can consider the graph $\Gamma$ given by $V=\{a,b,c,d\}$ and $E=\{(a,d),(b,d),(c,d)\}.$ Its geometric realization is a tripod in the plane. If we consider the collections
 \begin{eqnarray*}
 T_1 & =& \{\{(a,d)\},\{(b,d)\},\{(c,d)\}\} \\
 T_2 & = &\{\{(a,d),a\},\{(b,d),b\},\{(c,d),c\},\{(a,d),(b,d),(c,d),d\}\}
 \end{eqnarray*}
 The collection $\tau=2^{T_1\cup T_2}$ is a topology for $\Gamma.$ Later we will see that this topology is related to the metric space structure on the graph.
\end{example}

\subsection{Exercises:}

\begin{enumerate}

\item Could the empty space be considered a topological space? In the case of a positive answer, describe a topology on it. 

\item Consider $X=\{0,2,4,6,8\}$ and $\tau=\{\varnothing, \{0\},\{0,2\},\{0,4\},\{0,2,6\},\{0,4,6\},X\}.$ Is it $\tau$ a topology for $X$? Elaborate your answer (this means you prove all topology properties or give a counter-example for those properties)

\item With the same $X$ as the previous exercise. Give a topology for $X$ with no open sets with two elements.

\item Prove that the discrete topology and the co-finite topology on $X$ are the same. 

\item Prove that if $X$ is any finite set, then the discrete topology and co-finite topology are the same. 

\item Give a counter-example of the previous statement when $X$ is an infinite set. \emph{Consider an ordered set of numbers.}

\item Given the graph $\Gamma=\left(\{a,b,c\},\{(a,b),(b,c),(c,a)\}\right).$ Find the sets $T_1$ and $T_2$ that \emph{generate} its topology.

\item Given a set $X,$ a \emph{basis} for a topology of $X$ is a collection $\mathcal{B}$ of subsets of $X$ (called base elements) such that:
\begin{itemize}
\item For each $x\in X,$ there is at least one base element $B$ containing $x.$
\item If $x$ belongs to the intersection of the base elements $B_1$ and $B_2,$ then there is a base element $B_3$ containing $x$ such that $B_3\subset B_1\cap B_2.$ 
\end{itemize}
If $\mathcal{B}$ satisfies both conditions, we define the \emph{topology generated by} $\mathcal{B}$ as: a subset $U$ of $X$ is said to be open in $X$ (an element of $\tau$) if for each $x\in U$ there is a base element $B\in\mathcal{B}$ such that $x\in B$ and $B\subset U.$

Using the previous definition, prove that the open intervals in $\mathbb{R}$ are a basis for a topology on it. This topology is called the standard topology on $\mathbb{R}.$ $\star$
\end{enumerate}

\section{Topological Properties}



%\chapter{Second Chapter}
%
%\appendix % From here on wards, chapters are numbered with letters, as is the appendix convention
%
%\pagelayout{wide} % No margins
%\addpart{Appendix}
%\pagelayout{margin} % Restore margins
%
%\chapter{Some more blind-text}


%----------------------------------------------------------------------------------------

\end{document}