 % Load the kaobook class
\documentclass[
	fontsize=10pt, % Base font size
	twoside=false, % Use different layouts for even and odd pages (in particular, if twoside=true, the margin column will be always on the outside)
	%open=any, % If twoside=true, uncomment this to force new chapters to start on any page, not only on right (odd) pages
	secnumdepth=1, % How deep to number headings. Defaults to 1 (sections)
]{kaobook}

\usepackage[english]{babel} % Load characters and hyphenation
\usepackage[english=british]{csquotes}	% English quotes
\usepackage{kaobiblio}
%\addbibresource{minimal.bib}
\usepackage{kaotheorems}

% Load the package for hyperreferences
\usepackage{kaorefs}

\graphicspath{{images/}{./}} % Paths where images are looked for

\makeindex[columns=3, title=Alphabetical Index, intoc]
\begin{document}

%----------------------------------------------------------------------------------------
%	BOOK INFORMATION
%----------------------------------------------------------------------------------------

\titlehead{A manual for MA2007B}
\title[Applications of Geometry and Topology for Data Science]{Applications of Geometry and Topology for Data Science}
\author[AUP]{Alejandro Ucan-Puc}
\date{\today}
\publishers{Instituto Tecnólogico y de Estudios Superiores de Monterrey}

\maketitle
%----------------------------------------------------------------------------------------

\frontmatter % Denotes the start of the pre-document content, uses roman numerals
%----------------------------------------------------------------------------------------
%	PREFACE
%----------------------------------------------------------------------------------------

%\chapter*{Preface}


%----------------------------------------------------------------------------------------
%	TABLE OF CONTENTS & LIST OF FIGURES/TABLES
%----------------------------------------------------------------------------------------

\begingroup % Local scope for the following commands

% Define the style for the TOC, LOF, and LOT
%\setstretch{1} % Uncomment to modify line spacing in the ToC
%\hypersetup{linkcolor=blue} % Uncomment to set the colour of links in the ToC
\setlength{\textheight}{230\vscale} % Manually adjust the height of the ToC pages

% Turn on compatibility mode for the etoc package
\etocstandarddisplaystyle % "toc display" as if etoc was not loaded
\etocstandardlines % "toc lines as if etoc was not loaded

\tableofcontents % Output the table of contents

\listoffigures % Output the list of figures

% Comment both of the following lines to have the LOF and the LOT on different pages
\let\cleardoublepage\bigskip
\let\clearpage\bigskip

\listoftables % Output the list of tables

\endgroup

%----------------------------------------------------------------------------------------
%	MAIN BODY
%----------------------------------------------------------------------------------------

\mainmatter % Denotes the start of the main document content, resets page numbering and uses arabic numbers
\setchapterstyle{kao} % Choose the default chapter heading style

\chapter{Introduction to Topology}

%\pagelayout{wide} % No margins
%\addpart{aasja}
%\pagelayout{margin} % Restore margins

Topology was first studied in...

\section{Basic concepts and Examples}

Topology is the study of properties that are preserved under continuous functions in a generic way, for this reason, we need to work with objects that are proper to the space and these objects are subsets. The set theory is needed to properly understand topology because it is the main language of the theory, if you need a refresh of set theory you can see the reference.

\begin{definition}[Topological Space]
Let $X$ be a set and $\tau$ a collection of subsets of $X.$ We say that $\tau$ is a \emph{topology} for $X$ is it satisfies:
\begin{enumerate}
\item If the total and the empty set belong to $\tau.$  $$\varnothing, X\in\tau.$$

\item For every subcollection of elements of $\tau,$ let say $\{U_\alpha\}_{\alpha\in A},$ we have that its union is an element of $\tau.$ $$\bigcup_{\alpha \in U} U_\alpha \in \tau.$$

\item For every finite sub-collection of elements of $\tau,$ let say $\{U_j\}_{j=1}^n,$ we have that its intersection is an element of $\tau.$ $$ \bigcap_{j=1}^n U_j\in\tau.$$
\end{enumerate}

The elements of $\tau$ are called \emph{open sets} of $X$ and the pair $(X,\tau)$ is called a \emph{topological space}.
\end{definition}

This is not the first time that we see the concept of openness, for example in calculus we already worked with open intervals, open boxes (the real plane or space), and open disks (complex numbers). Topology theory intends to generalize this concept to general sets.

\begin{example}[Trivial topology]
Let $X$ be any set and consider the collection $\tau=\{\varnothing, X\}.$ $\tau$ is a topology for $X$ and it is called the trivial topology. 

This is not the best topology to understand properties of $X$ but it is a good counterexample when you want to generalize topological constructions.
\end{example}

\begin{example}[Discrete topology]
Let $X$ be any set and consider the collection $2^X=\{\mbox{all possible subsets of } X\}.$ $\tau$ is a topology for $X$ and it is called the discrete topology. 

Note that each element of $X$ is an open set on this topology. 
\end{example}

\begin{example}[Co-finite topology]
Let $X$ be any set and consider the collection $\tau_{cf}=\{U\subset X: X\setminus U \mbox{ is finite or }\varnothing\},$ i.e., the complement of $U$ in $X$ is a finite set of points. $\tau$ is a topology for $X$ and it is called the co-finite topology. 
\end{example}


\begin{example}[Spaces with different topologies]
Let $X=\{1,3,5,7\}$ and consider the collections

\begin{eqnarray*}
\tau_1 &=& \{\varnothing, \{1\},\{5\},\{7\},\{1,5\},\{1,7\},\{5,7\},\{1,5,7\},X\}. \\
\tau_2 &=& \{\varnothing, \{1\},\{3\},\{7\},\{1,3\},\{1,7\},\{3,7\},\{1,3,7\},X\}.
\end{eqnarray*}

Both $(X,\tau_1)$ and $(X,\tau_2)$ are topological spaces but they are not the same space. For example, the open set $\{1,5\}$ is not part of $(X,\tau_2)$ and $\{1,3\}$  is not part of $(X,\tau_1).$ 

Therefore in order to obtain the same topological space we need to find exactly the same open sets in both topologies.
\end{example}

The next example states a definition that will be useful in future chapters (see Simplicial Complexes and Persistent homology). 

\begin{example}[Graphs topology]
A \emph{graph} is a combinatorial object that consists of two sets:
\begin{itemize}
\item $V$ called vertices, and usually you can consider it as a set of points.
\item $E$ called edges, it is a subset of $V\times V$ (pairs of elements in $V$). This set describes how to vertices are related.
\end{itemize}
 
 Usually, graphs are denoted as $\Gamma=(V, E).$ Given a graph, you can associate a geometric object as a set of points in $\mathbb{R}^n$ in identification with $V$ and lines joining this set of points with the information of $E.$ This geometric object is called the \emph{geometric realization}.
 
% \begin{figure}[h]
 %put examples of graphs 
 %\end{figure}
 
 For example, we can consider the graph $\Gamma$ given by $V=\{a,b,c,d\}$ and $E=\{(a,d),(b,d),(c,d)\}.$ Its geometric realization is a tripod in the plane. If we consider the collections
 \begin{eqnarray*}
 T_1 & =& \{\{(a,d)\},\{(b,d)\},\{(c,d)\}\} \\
 T_2 & = &\{\{(a,d),a\},\{(b,d),b\},\{(c,d),c\},\{(a,d),(b,d),(c,d),d\}\}
 \end{eqnarray*}
 The collection $\tau=2^{T_1\cup T_2}$ is a topology for $\Gamma.$ Later we will see that this topology is related to the metric space structure on the graph.
\end{example}

\subsection{Exercises:}

\begin{enumerate}

\item Could the empty space be considered a topological space? In the case of a positive answer, describe a topology on it. 

\item Consider $X=\{0,2,4,6,8\}$ and $\tau=\{\varnothing, \{0\},\{0,2\},\{0,4\},\{0,2,6\},\{0,4,6\},X\}.$ Is it $\tau$ a topology for $X$? Elaborate your answer (this means you prove all topology properties or give a counter-example for those properties)

\item With the same $X$ as the previous exercise. Give a topology for $X$ with no open sets with two elements.

\item Prove that the discrete topology and the co-finite topology on $X$ are the same. 

\item Prove that if $X$ is any finite set, then the discrete topology and co-finite topology are the same. 

\item Give a counter-example of the previous statement when $X$ is an infinite set. \emph{Consider an ordered set of numbers.}

\item Given the graph $\Gamma=\left(\{a,b,c\},\{(a,b),(b,c),(c,a)\}\right).$ Find the sets $T_1$ and $T_2$ that \emph{generate} its topology.

\item Given a set $X,$ a \emph{basis} for a topology of $X$ is a collection $\mathcal{B}$ of subsets of $X$ (called base elements) such that:
\begin{itemize}
\item For each $x\in X,$ there is at least one base element $B$ containing $x.$
\item If $x$ belongs to the intersection of the base elements $B_1$ and $B_2,$ then there is a base element $B_3$ containing $x$ such that $B_3\subset B_1\cap B_2.$ 
\end{itemize}
If $\mathcal{B}$ satisfies both conditions, we define the \emph{topology generated by} $\mathcal{B}$ as: a subset $U$ of $X$ is said to be open in $X$ (an element of $\tau$) if for each $x\in U$ there is a base element $B\in\mathcal{B}$ such that $x\in B$ and $B\subset U.$

Using the previous definition, prove that the open intervals in $\mathbb{R}$ are a basis for a topology on it. This topology is called the standard topology on $\mathbb{R}.$ $\star$
\end{enumerate}

\section{Topological Properties}

In mathematics, each research field is conformed by a pair: the objects and the transformations, so we are interested in the properties of the objects that are preserved by the transformations. In the previous section, we introduce topological spaces, that correspond to the objects of topology. Lately, we will introduce the transformations, but now we will present a set of properties of topological spaces that will be of our interest during the study of data.


\subsection{Closedness}

We mentioned that topology is the study of spaces using a collection of subsets called: open sets. The first property is to complement the open sets. 

\begin{definition}	
Given a topological space $(X,\tau)$ and a subset $V$ of $X.$ We said that $V$ is closed if its complement in $X$ is an open set, i.e., $$X\setminus V:= \{x\in X: x\not\in V\}\in \tau.$$
\end{definition}

Closed sets on a topological space form a key part of topology, they are related to other topological properties and theorems. For example, closed sets are present in (topological) metric spaces to define continuity and compactness (we will see it later). Also, we can find closed sets in the definition of separation properties. 

\begin{example}
Assume that $X=\{1,3,5,7\}$ with the topology $\tau=\{\varnothing,\{1\},\{5\},\{1,5\},\{1,5,7\},X\}.$ 

In this topological space, the sets $\varnothing, \{1,\},\,\{3,7\},\{3,5,7\}, X,$ are closed sets. Even more, we can enumerate all closed sets of $X.$ 
 
\end{example}

Note that, in a topological space, some sets could have open and closed properties. In the previous example, $X$ is open and closed at the same time. In the literature, this sets are called \emph{clopen} sets, but on this notes we will not give much attention on these.

\subsection{Connectedness}

As its name indicates, connectedness indicates if a set is made by part or it is a unit. Let detail on this. 

\begin{definition}
Given a set $V$ of a topological space, a partition of $V$ is a collection of open sets $\{A\}_j$ such that $A_i$
\end{definition}

\begin{definition}
Given a set $V$ of a topological space $X,$ we said that $V$ is \emph{not connected} if there exists open sets $A$ and $B,$ such that:
\begin{enumerate}
\item $V=A\cup B,$
\item $A\cap B =\varnothing.$
\end{enumerate}

On the contrary, if there is no such partition, then we say that $V$ is \emph{connected}
\end{definition}

\begin{example}
Assume that $X=\{1,3,5,7\}$ with the topology $\tau=\{\varnothing,\{1\},\{5\},\{1,5\},\{1,5,7\},X\}.$ 

The set $\{1,5\}$ is not connected, because $\{1,5\}=\{1\}\cup\{5\}.$ The sets $\{1\},\{5\},\{1,5,7\}, X$ are connected. 
\end{example}

\begin{example}
Assume that $X=\{1,3,5,7\}$ with the discrete topology $\tau=2^X.$ On this topology, $X$ is no longer connected,  because: $$X=\{1\}\cup\{3,5,7\}=\{1,3\}\cup\{5,7\}.$$

In fact, the only connected sets are $\{1\},\{3\},\{5\}$ and $\{7\}.$
\end{example}

So connectedness of a space depends on the topology of the ambient space. 

\subsubsection{Connected compontents}

Between the connected subspaces of a topological space, there are some that are characteristic of the space. Before we state the definition of connected components, we need to introduce the concept of equivalence relations.

In mathematics, we can relate objects by equivalence relations. In mathematics, a \emph{relation} between two sets $X$ and $Y$ is just a set of ordered pairs $(x,y),$ for example a function is a relation. The ``lower or equal'' ($\leq$) is a relation between real numbers.

\begin{definition}
An equivalence relation on a set $X,$ denoted by $\sim,$ is a mathematical relation that satisfies:
\begin{enumerate}
\item Reflexive: each element is related to itself $$a\sim a.$$

\item Symmetric: if $a$ is related to $b,$ then $b$ is related to $a.$ $$a\sim b\Rightarrow b\sim a.$$

\item Transitivity: if $a$ is related to $b,$ and $b$ is related to $c,$ then $a$ is related to $c.$ $$a\sim b \mbox{ and }b\sim c\Rightarrow a\sim c.$$
\end{enumerate}
\end{definition}

What does an equivalence relation imply on a set? Well, given an equivalence relation on a set, we can ``collect'' all interrelated elements in a subset of the ambient set. These collections of interrelated elements are called \emph{equivalence classes} and they produce a partition on the ambient set.

 \begin{definition}
 Let $X$ be a set with an equivalence relation $\sim.$ We define the equivalence class of $x$ as the set $$[x]:=\{y\in X: y\sim x\}.$$
 
 Given the equivalence relation $[y],$ we say that $y$ is a class representative. 
 \end{definition}
 
 Examples of equivalence relations can be found in.
 
 We will define an equivalence relation on topological spaces given by connected sets.
 
 \begin{definition}
 Let $(X,\tau)$ be a topological space. Let $\sim$ defined as follow, $x$ and $y$ are related ($x\sim y$) if and only if there exists a connected subspace of $X$ containing both points. 
 
 The equivalence classes for this equivalence relation are called \emph{Connected components} of $X.$
 \end{definition}
 
\begin{example}
Assume that $X=\{1,3,5,7\}$ with the topology $\tau=\{\varnothing,\{1\},\{5\},\{1,5\},\{1,5,7\},X\}.$ 

The unique connected component of $X$ is $X$ itself, and this follows because the space is connected on this topology.
\end{example}

\begin{example}
Assume that $X=\{1,3,5,7\}$ with the discrete topology $\tau=2^X.$ On this topology, $X$ is no longer connected, and the connected components are $\{1\},\{3\},\{5\}$ and $\{7\}.$
\end{example}

\subsubsection{Connectedness on Graphs}

As graphs are an essential part of TDA, we need to understand how it works connectedness over graphs. 

Let $\Gamma=(V, E)$ be a graph, and we will think that there is no distinction between the combinatorial object and the geometric realization. 

\begin{definition}
We say that $\Gamma$ is \emph{path-connected} if every two points can be joined by a trajectory of continuous edges. 
\end{definition}

\begin{proposition}	
Let $\Gamma=(V, E)$ be a path-connected graph, then $\Gamma$ is connected in the graph topology. 
\end{proposition}
\begin{proof}
Assume that $\Gamma$ is not path-connected. Then there exists at least a pair of points that are not connected by a path. 
\end{proof}

For graphs, the connected components look like subgraphs (subsets that are graphs) that are path-connected.

\begin{example}
Consider the graph $K_n$ given by $V=\{1,2,\cdots, n\}$ and $E=V\times V.$ By definition, this graph is path-connected for every $n.$ 
\end{example}

\begin{example}
Consider the graph given by $V=\{1,2,\cdots, n\}$ and $E$ given by these conditions:
\begin{itemize}
\item $(v,u)\in E$ if $u$ and $v$ are even.
\item $(v,u)\in E$ if $u$ and $v$ are odd.
\end{itemize}

By its definition, this graph is not path-connected, and it is composed of two connected components.
\end{example}

\subsection{Compactness}

As you can imagine, the compactness observes if a set can be set 


\begin{definition}
Let $(X,\tau)$ be a topological space and $E$ be a subset. An open cover for $E$ is a collection of open sets of $X,$ $\{U_\alpha\}_{\alpha\in A},$ such that $$E\subset \bigcup_{\alpha\in A} U_\alpha.$$
\end{definition}

\begin{definition}
For $(X,\tau)$ and $E$ be a subset. We said that $E$ is \emph{compact} if for every open covering for $E$ there exists a finite cover (finite number of elements) that covers $E.$
\end{definition}

\begin{example}
Consider the set of positive integers $X=\mathbb{N}$ with the discrete topology $\tau=2^{\mathbb{N}}.$ Every finite set is compact. 

Also, if $X$ has the co-finite topology, then also every finite set is compact.
\end{example}

Now, the compactness property seems artificial and that maybe all subsets are compact but this does not happen. In the next section, we will introduce more examples of all properties on metric spaces.

\section{Metric Spaces and Metric Topology}

Metric spaces are an important set of examples for topology, this follows because their metric structure induces a natural topology on them.

\begin{definition}
Let $X$ be a non-empty set. A \emph{metric} on $X$ is defined as a function $d:X\times X\to \mathbb{R}$ that satisfies:
\begin{enumerate}
\item Non degeneracy: $d(p,q)=0$ if and only if $p=q.$
\item Symmetry: $d(p,q)=d(q,p).$
\item Triangle inequality: for every $p,q,r\in X,$ $$d(p,q)\leq d(p,r)+d(r,q).$$
\end{enumerate}

A \emph{metric space} is a pair $(X,d)$ where $d$ is a metric for $X.$
\end{definition}

\begin{example}
Let $X=\mathbb{R}$ and $d(x,y)=|x-y|,$ the function given by the absolute value. Then $(X,d)$ is a metric space. The metric $d$ is known as the standard metric on the reals.

We can consider the functions:
\begin{itemize}
\item $d_1(x,y)=\left|\ln\left(\frac{x}{y}\right)\right|.$

\item $d_2(x,y)=\frac{|x-y|}{1+|x-y|}.$
\end{itemize}

Both functions are metrics for $X.$
\end{example}

\begin{example}
Let $X=\mathbb{R}^n$ be the set of ordered $n-$tuples of reals. If $x=(x_1,\cdots,x_n)$ and $(y_1,\cdots,y_n),$ then let $$d(x,y)=\sqrt{\sum_{j=1}^n (x_j-y_j)^2}$$ this function makes $X$ a metric space, and it is known as the $\ell_2-$metric.
\end{example}
%\chapter{Second Chapter}
%
%\appendix % From here onwards, chapters are numbered with letters, as is the appendix convention
%
%\pagelayout{wide} % No margins
%\addpart{Appendix}
%\pagelayout{margin} % Restore margins
%
%\chapter{Some more blind-text}


%----------------------------------------------------------------------------------------

\end{document}